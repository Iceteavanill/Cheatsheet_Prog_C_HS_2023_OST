\section{Datentypen}

Datentypen legen fest,

\begin{itemize}[itemsep=1pt, parsep=0pt]
    \item wie lang das Bitmuster an der zugehörigen Speicherstelle ist,
    \item was dieses Bitmuster bedeutet.
\end{itemize}

\subsection{simple Ganzzahltypen}

\begin{itemize}[itemsep=1pt, parsep=0pt]
    \item \textbf{char} \\ immer 8 Bit (Als einziger Typ \textbf{IMMER} 1 Byte)
    \item \textbf{short} \\ für kleine ganzzahlige Werte (mind. 16 Bit)
    \item \textbf{int} \\ effizienteste Grösse für Prozessor (mind. 16 Bit)
    \item \textbf{long} \\ für grosse ganze werte (mind. 32 Bit)
    \item \textbf{long long} \\ für sehr grosse ganze werte (mind. 64 Bit)   
    \item \textbf{size\_t} \\ mindestens so gross wie die maximale Anzahl Adressen, nie negativ
\end{itemize}

Ganzzahltypen können \textbf{signed} (Vorzeichenbehaftet) oder \textbf{unsigned} sein. Wenn nicht definiert, meist signed ausser Datentyp \textbf{char}, dieser ist nicht standardisiert.\\
\textbf{Überlauf ist nicht definiertes Verhalten}\\
Der \textbf{unsigned} Wertebereich befindet sich von 0 bis $2^n -1 $\\
Der \textbf{signed} Wertebereich (Zweierkomplement) befindet sich von $-2^{n-1}$ bis $2^{n-1} -1 $\\
(n = anzahl bits)

\nextcol

\subsection{Stdint.h}

Stdint stellt vordefinierte Typen zur Verfügung welche Plattform unabhängige feste Grössen haben. z.B. :

\begin{itemize}[itemsep=1pt, parsep=0pt]
    \item int8\_t = signed 8 Bit
    \item uint64\_t = unsigned 64 Bit
    \item .....
\end{itemize}

Allerdings sind diese auf dem Zielsystem möglicherweise nicht sehr effizient, da sie die native Busbreite über oder unterschreiten. 
Ein Int ist der effizienteste Datentyp.

\subsection{Gleitpunktzahl}

Gleitpunktzahlen speichern Zahlen wissenschaftlicher Schreibweise. 
Dabei treten allerdings fast immer Rundungsfehler auf (\textbf{nie auf Gleichheit testen}) und brauchen viel Rechenleistung. 
Sie können aber auch Werte wie $\pm$ inf, $\pm$ 0, NaN (not a number) darstellen.\\
Speicherzusammensetzung, Standardisierter float (IEEE 754): 
\begin{center}
    \begin{tabular}{|c|c|c|c|c|} \hline  
        (in bit) & Sign  & Exponent & Mantisse & total\\ \hline  
        float & 1 & 8 & 23 & 32 bits\\ \hline  
        double & 1 & 11 & 52 & 64 bits\\ \hline   
    \end{tabular}
\end{center}

Wenn eine Zahl mit einem Koma (z.B: 12.91) definiert wird, ist sie automatisch eine Gleitpunktzahl. 
Dieser Effekt kann verwendet werden, um bei eine Mathematischen Berechnung die Genauigkeit zu verbessern, wenn mit Ganzzahltypen gerechnet wird. 
Wenn eine Zahl implizit als Gleitpunktzahl (Möglichst am Anfang der Rechnung) definiert wird, wird implizit mit einer Gleitpunktzahl weiter gerechnet und am Schluss wieder zu einer Ganzzahl umgewandelt.

\subsection{Enum}

\subsubsection{Aufzählungstyp}

Mit einem Enum (im code immer klein schreiben) kann ein Aufzählungstyp erreicht werden für z.B. für state machines. Generelle Syntax:\\
\verb|enum type-name {item1 = value1, item2 = value2, …};|\\
Die Value Deklaration kann ausgelassen werden. 
Der Compiler vergibt von 0 auf aufzählend automatisch Werte. 
Es kann strategisch eine Zahl dazwischen gesetzt werden, um gewisse numerische Zahlen zu Erhalten.\\
Wenn man ein Enum verwenden will, muss das Schlüsselwort \say{Enum} immer mitgeschrieben werden. 
Mit einer Typendeklaration kann das vereinfacht werden. 

\lstinputlisting{code/enum.c}

\nextcol

\subsubsection{Ganzzahkonstanten}

Man kann per Enum auch Ganzzahlkonstanten definieren. Sie sind sicherer als ein \verb|#define| und erlauben auch das Definieren von grössen eines Arrays. 
Definition:

\lstinputlisting{code/ganzzahl.c}

\subsection{Pointer}\label{Pointerbasics}

Pointer sind Variablen welche eine Speicheradresse enthalten. 
Sie sind mindestens so gross, dass alle Speicheradressen definiert werden können.\\
Wenn Pointer Referenziert (Operator : \verb|&|) werden, liest man die Adresse, auf die der Pointer zeigt.\\
Wenn Pointer dereferenziert (Operator : \verb|*|) werden, liest man den Wert, auf der gezeigt wird.\\ 
Pointer müssen auch den Variabeltyp, auf den sie Zeigen, definiert bekommen um das Bitmuster auf das sie zeigen Interpretieren zu können. 
Ausnahme davon sind die \textbf{Voidpointer}. 
Ihnen kann jeder Pointertyp übergeben werden. 
Diese dürfen aber \textbf{nicht} dereferenziert werden.\\
Wenn ein Pointer nicht gebraucht wird, muss er \say{\textbf{NULL}} (nicht zwingend numerisch 0) gesetzt werden. 
Das Dereferenzieren eines NULLpointer  führt zu undefined behaviour.\\

\lstinputlisting{code/pointer_basics.c}

\subsubsection{Pointer Arithmetik}

\begin{itemize}[itemsep=1pt, parsep=0pt]
    \item Pointer unterschiedlicher Datentypen dürfen einander nicht zugewiesen werden (Schutzmechanismus)
    \item Einem Pointer eines bestimmten Typs dürfen Pointer dieses Typs oder void-Pointer zugewiesen werden
    \item Einem void-Pointer dürfen beliebige Pointer zugewiesen werden (nützlich, aber gefährlich)
\end{itemize}
Auf Pointer darf des weiteren addiert und subtrahiert werden. Ganze zahlen verschieben ihn dabei jeweils um ganze Elemente. 
Daher kann man nicht zwischen 2 Elemente Zeigen. Darum ist der Pointer Typ wichtig. 
Es ist keine Arithmetik auf einen Voidpointer möglich.\\
Pointer darf man auch mit anderen Pointer vergleichen(==, !=,...).

\nextcol

\subsubsection{Memorymap}

Eine Memorymap wird verwendet, um den Zustand von Pointer und Variablen zu signalisieren. 
Das folgende Beispiel zeigt den Zustand von der Deklaration von \ref{Pointerbasics}.\\
\begin{center}
    \ifthenelse{\boolean{svgWorks}}{
        \includesvg[width = 0.7\linewidth, inkscapelatex=false]{svg/Pointer_Memorymap.svg}    
    }{
        \includegraphics[width=.5\columnwidth]{Pictures/Pointer_memory_map.png}
        }
\end{center}

\subsection{Deklarieren}

Generelle Syntax : \verb|<Datentyp> <Variablenname>;|\\
Erlaubte Zeichen sind Buchstaben, Ziffern und Underscore(\_). Eine Zahl darf nicht zu beginn stehen. Umlaute sind nicht erlaubt. Variablennamen sind \textbf{case sensitive!}\\
Beispiele:

\lstinputlisting{code/deklaration.c}

Es gibt Namen, die nicht erlaubt sind, da sie vom C Standard reserviert sind. Diese sind:\\

{\tabcolsep=0.1cm
\resizebox{\columnwidth}{!}{%
\begin{tabular}{cccccc}
auto             & break           & case      & char      & const       & continue   \\
default          & do              & double    & else      & enum        & extern     \\
float            & for             & goto      & if        & inline      & int        \\
long             & register        & restrict  & return    & short       & signed     \\
sizeof           & static          & struct    & switch    & typdef      & union      \\
unsigned         & void            & volatile  & while     & \_alignas   & \_alignof  \\
\_Atomic         & \_Bool          & \_Complex & \_Generic & \_Imaginary & \_Noreturn \\
\_Static\_assert & \_Thread\_local &           &           &             &           
\end{tabular}%
}
}

\subsection{Umwandeln}

Es gibt die implizite und explizite Umwandlung. Implizit erledigt der Compiler da er selbst ein missmatch von Typen erkennt $\rightarrow$ schlecht da es eine Warnung erzeugt.\\
Explizit wird durch Schreiben des Typs neben der umzuwandelnden Zahl erreicht.

\lstinputlisting{code/umwandeln.c}

\nextcol

\subsection{Scope}

\begin{itemize}[itemsep=1pt, parsep=0pt]
    \item \textbf{global} \\ nicht in einer Funktion deklariert, 0 initialisiert, Laufzeit des Programms, immer sichtbar
    \item \textbf{lokal} \\ in Funktion deklariert, nicht initialisiert, Laufzeit der Funktion, nur in der Funktion Sichtbar, auch wenn eine neue Funktion aufgerufen wird
    \item \textbf{static} \\ Ausnahme $\rightarrow$ kommt auf Anwendung an. (siehe Static$\rightarrow$ \ref{static})
\end{itemize}

\subsection{Arrays}

Ein Array bietet eine kompakte Zusammenfassung von mehreren Variablen des gleichen Basistyps. 
Array Nummerierung beginnen immer bei 0! 
Sie sind so lang wiebei der Deklarierung definiert. 

\lstinputlisting{code/array_basics.c}

\subsubsection{Initialisierung}

Man kann ein Array entweder komplett oder gar nicht initialisieren.

\lstinputlisting{code/array_init.c}

\subsubsection{Char Arrays}

Mit Char Arrays werden Zeichenketten realisiert. Char Arrays müssen immer NULL('\textbackslash 0') terminiert werden, wenn mit diesen Standardfunktionen zur Char Array Verarbeitung verwendet werden. Die Nullterminierung wird verwendet, um zu wissen wann die Zeichenkette fertig ist.

\lstinputlisting{code/array_char.c}

\subsubsection{Memorymap}

Das Format der Memorymap für Arrays sieht folgendermassen aus:\\

\begin{tabular}{|c|c|c|c|c|}
\hline
Wert 0&Wert 1& Wert 2& Wert 3&Wert 4\\ \hline
\end{tabular}
\\

Unter den jeweiligen Werten muss die Element Nummer Stehen.

\nextcol

\subsubsection{Handhabung von Arrays}

Arrays können nicht direkt miteinander verglichen werden, in Funktionen direkt übergeben werden oder direkt auf ein anderes überspielt werden (memcpy kann verwendet werden).\\
Bei einem Funktionsaufruf wird ein Pointer auf das erste Element des Arrays übergeben.

\subsubsection{Mehrdimensionale Arrays}

Mehrdimensionale Arrays können so verstanden werden das ein Array Element wieder ein ganzes Array enthaltet.\\
Bei einem 2D Array kann man sich eine Matrix vorstellen, wobei beim Aufruf [Zeile][Spalte] mitgegeben wird.\\
Definition eines mehrdimensionalen Arrays:\\
\noindent
\begin{minipage}{0.5\columnwidth}
\lstinputlisting{code/mehrdim_array.c}
\end{minipage}
\begin{minipage}{0.5\columnwidth}
\begin{tabular}{|c|c|c|c|}
\hline
1 & 2 & 3 & 4 \\ \hline
5 & 6 & 7 & 8 \\ \hline
9 & 0 & 1 & 2 \\ \hline
\end{tabular}
\end{minipage}

Ebenso könnte man alpha so initialisieren:

\lstinputlisting{code/array_mehrdim_init.c}

\subsection{Pointer und Arrays}

Pointer und Arrays können ähnlich verwendet werden. 
In aufgerufenen Funktionen kann mit einem Pointer so wie mit einem Array weitergearbeitet werden. 

\lstinputlisting{code/array_poiner.c}

\subsubsection{Char Array}

Eine String Definition kann auch so erfolgen:

\lstinputlisting{code/array_pointer_char.c}

\subsection{Typ-attribute}

\subsubsection{const}

Das Attribut const (vor dem Variablentyp geschrieben) markiert eine Variable als read only für den Quellcode, die Hardware kann aber dennoch den Wert verändern.\\
Ein Const Array besteht nur aus konstanten.\\
Für ein konstanten Pointer muss const nach dem * stehen. Sonst ist nur der Wert auf den der Pointer zeigt read only. 

\lstinputlisting{code/typatribute.c}

Um sicherzugehen was wirklich konstant ist kann die Definition von rechts nach links gelesen werden. Was links neben einem const steht ist wirklich const.\\

In Funktionsköpfen lohnt es sich const als Sicherheit einzubauen, so das nicht versehentlich eine variable geschrieben wird.

\lstinputlisting{code/funktionskopf.c}

\subsubsection{Volatile}

Volatile signalisiert dem Compiler das sich die Variable auch ausserhalb der normalen Programmausführung ändern kann. Einsatz bei Hardware naher Programmierung und multithreading.

\subsubsection{static} \label{static}

Static kann eine Variabel und eine Funktion sein, diese haben aber jeweils verschiedene Bedeutungen:
\begin{itemize}
    \item Globale Variabel oder Funktion:\\
    Sie sind nur in der Compile Unit gültig in der sie definiert sind.
    \item lokale Variabel:\\
    haben dieselbe \textbf{Lebensdauer wie eine Globale Variabel}, Sichtbarkeit auf Funktionsblock beschränkt und mit 0 oder einer konstanten  einmalig initialisiert. 
\end{itemize}

\subsection{Structs}

Mittels Struct kann man verschiedene Datentypen zu einem Typ zusammenführen. Auch Structs kann man in einen anderen Struct verwenden.\\

\noindent
\begin{minipage}{0.5\columnwidth}
Definition:
\lstinputlisting{code/struct_def.c}
\end{minipage}
\begin{minipage}{0.5\columnwidth}
bsp:
\lstinputlisting{code/struc_bsp.c}
\end{minipage}

\subsubsection{Beispiel mit Pointer}

Das Beispiel verwendet Deklarationen von oben

\lstinputlisting{code/struc_pointer.c}
\subsubsection{Grösse}

Die Grösse eines Struct ist nicht immer die Summe aller Typen. Aufgrund von Hardwarelimitationen (Alignment) können padding bytes in der Struct eingesetzt werden welche einfach leer sind. Daher kann man die wirkliche Grösse nur durch sizeof herausfinden.

\subsubsection{Funktionen}

Funktionsaufrufparameter und Funktionsreturns können Structs sein. Allerdings ist dies nicht effizient und eine Übergabe des Pointers wäre besser da in diesem Fall auch Arrays direkt üergeben werden.

\subsection{Union}

Unions haben eine identische Syntax wie Structs, allerdings teilen \textbf{alle} Elemente den \textbf{selben} Speicherbereich. Man kann es verwenden, um z.B. einfacher an die einzelnen Bytes einer IpV4 Adresse zu kommen.

\lstinputlisting{code/union.c}

\subsection{typedef}

Man kann sich auch eigene Typennamen erstellen mit typedef.\\
syntax : \verb|typedef  <Typ> <Name des neuen typ>|\\
stdint.h verwendet dies. 
Auch sehr nützlich um das Schlüsselwort Enum/Struct zu sparen. 
\lstinputlisting{code/typdef.c}