\section{Compiler}
Der Compiler übersetzt den Quellcode zu Machinecode welcher durch das Zielsystem ausführbar ist.
\subsection{Molularisierung}

Der gesamte Code wird auf mehrere Module aufgeteilt, um die Übersichtlichkeit und Wiederverwendbarkeit zu verbessern. Ein Modul besteht in der Regel aus einer .h- und einer .c-Datei.\\
Die Schnittstelle des Moduls wird in der .h-Datei (Header) definiert und dokumentiert. Sie enthält: Prototypen für alle Funktionen, welche die Schnittstelle nach aussen bilden, Dokumentation wie diese Funktionen benutzt werden, structs, enums, unions, typedef, Präprozessormakros. Die Implementationen sind in der .c Datei

\subsubsection{Information hiding}

Information hiding bedeutet: nur so viele Informationen nach Aussen geben wie nötig. Daher globale Variablen sparsam verwenden. Diese Strategie macht Code sicherer und weniger fehleranfällig.

\subsubsection{Richtiges Inkludieren}

\begin{lstlisting}[language = c]
#include "eigenesModul.h"//eigenes Modul
#include <stdio.h>//standartlibrary
\end{lstlisting}


Um mehrfachincludes zu verhindern, gibt es sogenannte Includeguards
\begin{lstlisting}[language = c]
#ifndef switch_h
#define switch_h
... //implementation
#endif

#pragma once//moderne Variante
\end{lstlisting}

\subsection{Build Prozess}

Der Build Prozess erzeugt ein ausführbares Programm und besteht wesentlich aus :
\begin{itemize}[itemsep=1pt, parsep=0pt]
    \item jede .c-Datei compilieren (erzeugt .o Dateien)
    \item jede erzeugte .o-Datei linken
\end{itemize}


\subsection{Präprozessor}

Der Präprozessor kann einfache Textersetzung machen und so z.B. konstanten einsetzen.\\
Wichtige makros sind dabei:

\lstinputlisting{code/preprocessor.c}

Allerdings sind solche Präprozessormakrofunktionen sehr fehleranfällig da sie nur Textersetzung machen.

\subsection{inline}

Das Schlüsselwort inline ermöglicht es dem Compiler, den Funktionsaufruf komplett wegzuoptimieren. Es gibt jedoch keine Garantie, dass er dies auch tut! Es wird häufig zusammen mit static verwendet.\\

\lstinputlisting{code/inline.c}

Das Beispiel hat die selbe Funktion wie die des Präprozessors, allerdings ohne Potenzial für undefined behaviour.

\subsection{Anwendung des compilers}

Es wird angenommen das clang verwendet wird:\\

Ein beispielhafter compilvorgang wäre:\\
\verb|clang -Wall -o test test.c|

\subsubsection{Clang flags}
\begin{itemize}[itemsep=1pt, parsep=0pt]
    \item \textbf{-Wall} Alle Fehler und Warnungen ausgeben.
    \item \textbf{\textless{}File.c/.o/.a\textgreater{}} bindet die Datei in den compile Prozess ein.
    \item \textbf{-o} \textless{}File\textgreater{}Spezifiziert der Name der Ausgabedatei und macht diese ausführbar.
    \item \textbf{-c} \textless{}File.c\textgreater{}Kompiliert die  .c Files zu .o Files. Diese sind noch nicht gelinkt. Diese müssen noch mit -o gelinkt werden.
    \item \textbf{-O3} Optimiert den Code auf Geschwindigkeit.
    \item \textbf{-Os} Optimiert den Code auf Grösse.
    \item \textbf{-I} \textless{}pfad\textgreater{} Pfad zu externen Libraries welche inkludiert werden sollen
    \item \textbf{-lm} falls math.h verwendet wird, muss zusätzlicher Machinencode verlinkt werden
\end{itemize} 