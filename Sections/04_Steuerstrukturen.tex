\newpage
\section{Steuerstrukturen}

\subsection{basics}

Die Wichtigen Steuerstrukturen in einer Auflistung.

\lstinputlisting{code/steuerstrukturen_basics.c}

\subsubsection{Sprunganweisungen}

\begin{itemize}[itemsep=1pt, parsep=0pt]
    \item \textbf{break} \\ (do)while, for, switch abbrechen.
    \item \textbf{continue} \\ bei (do)while und for in den nächsten Schleifendurchgang springen.
    \item \textbf{return} \\ aus Funktion an aufrufende Stelle zurückspringen
    \item \textbf{goto} \\ innerhalb einer Funktion an eine Marke (Label) springen (sind selten gerechtfertigt verwendet)
\end{itemize}

\subsubsection{GoTo}

Ein mit einem goto kann man \textbf{innerhalb einer Funktion} zu einer Marke Springen. Das kann sehr komfortabel sein, führt aber fast immer zu sehr schwer lesbaren Code. Möglichst nie verwenden.\\
Ein Beispiel:

\lstinputlisting{code/goto.c}

\subsection{Funktionen}

Funktionen sind eine Zusammenfassung von Anweisungen, welche ein und Ausgabewerte haben können.\\
Deklaration:\\
\verb|<returnType> <functionName>(<paramType1> <paramName1>,|\\
\verb| <paramType2> <paramName2>, …);|\\
Wenn eine Funktion aufgerufen wird, müssen immer \textbf{alle} Parameter mitgegeben werden.\\

\subsubsection{Funktionsprototyp}

Funktionen werden typischerweise weit oben im Quelltext definiert, als Funktionsprototyp. 
Der Funktionsprototyp legt das Interface der Funktion fest. 
Er \textbf{muss} verwendet werden, wenn die Funktion vor Definition verwendet werden möchte.

\begin{lstlisting}[language = c]
int myFunc(int input1, int input2,...);//Funktionsprototyp
int myfunc(int input1)//direkte definition
{
    return 0;
}
\end{lstlisting}
Der Parametername kann zwischen Prototyp und Definition unterschiedlich sein, ist aber nicht empfohlen.

\subsubsection{Void}

Void als Returntyp heisst das kein Wert zurückgegeben wird.\\
Return darf verwendet werden, aber es darf kein Wert hinterlegt werden.\\
Void als Parameter heisst das kein Parameter übergeben werden kann.

\nextcol

\subsection{Pointer auf Funktionen}

Pointer können auch auf Funktionen definiert werden zum z.B. zur Laufzeit dynamische Programmabläufe zu realisieren.\\
Syntax: int (*name)(int,int);

\begin{itemize}[itemsep=1pt, parsep=0pt]
    \item \textbf{int} : Return Typ der funktion
    \item \textbf{(*name)} : name des Pointer
    \item \textbf{(int,int)} : aufrufparameter(hier 2*int)

\end{itemize}

bsp:

\lstinputlisting{code/pointer_to_func.c}

\subsection{Funktionen : Iteration und Rekursion}

\begin{itemize}[itemsep=1pt, parsep=0pt]
    \item Rekursion: eine Funktion ruft sich selbst auf.
    \item Iteration: Algorithmus enthält Abschnitte, die innerhalb einer Ausführung mehrfach durchlaufen werden (Schleife)
    \item Die rekursive Form kann eleganter sein, ist aber praktisch immer ineffizienter als die iterative Form.
    \item Das Abbruchkriterium ist bei beiden Formen zentral
\end{itemize}

\subsubsection{Anwendungen}
\begin{itemize}[itemsep=1pt, parsep=0pt]

    \item Backtracking-Algorithmen z.B. Finden eines Weges durch ein Labyrinth (zurück aus Sackgasse und neuen Weg prüfen)
    \item Implementierung rekursiv definierter mathematischer Funktionen (z.B. Fakultätsfunktion)
    \item \textbf{Achtung}: der Speicherbedarf ist bei Rekursion schwer abzuschätzen. Rekursive Funktionen sind deshalb insbesondere bei der Embedded Programmierung zu vermeiden.
    \item \textbf{Achtung}:jede Instanz einer rekursiv gerufenen Funktion besitzt ihre eigenen, unabhängigen Variablenkopien. Ausnahme: static-deklarierte lokalen Variablen
    \item Indirekte Rekursion ist unbedingt zu vermeiden Diese kommt z.B. zustande, wenn sich zwei oder mehrere Funktionen wechselseitig oder im Kreis aufrufen
\end{itemize}

\nextcol

\subsubsection{Beispiel}

Hier diese Folgenden Funktionen berechnen beide die Fakultät einer Zahl.\\

\noindent
\begin{minipage}[t]{0.5\columnwidth}
Fakultät \textbf{Iterativ}
\begin{lstlisting}[language = c]
uint64 fak(uint64 n)
{
    uint64 val = 1;
    for (int i = 2; i <= n; ++i)
    {
        val = val * i;
    }
    return val;
}
\end{lstlisting}
\end{minipage}
\begin{minipage}[t]{0.5\columnwidth}
Fakultät \textbf{rekursiv}
\begin{lstlisting}[language = c]
uint64 fak(uint64 n)
{
    if (n > 1)
    {
        return n * fak(n-1);
    }//Erneuter Aufruf der Funktion fak
    else
    {
        return 1;
    }
}
\end{lstlisting}
\end{minipage}
\subsection{Operatoren}

Operatoren haben eine Priorität, welche aussagt in welcher Reihenfolge die Befehle ausgeführt werden. 
Von höchster Priorität zu niedrigster.\\
Haben Operatoren dieselbe Priorität besagt die Assoziativität in welcher Reihenfolge ausgewertet wird.\\


{\tabcolsep=0.05cm
\resizebox{\columnwidth}{!}{%
\begin{tabular}{|c|l|l|c|}
\hline
\textbf{Priorität}  & \textbf{Symbol}         & \textbf{Bedeutung}          & \textbf{Assoziativität} \\ \hline
\multirow{5}{*}{15} & (Postfix)  $++\quad--$  & Postfix-Ink/dekrement       & \multirow{5}{*}{L - R}  \\
                    & ()                      & Funktionsaufruf             &                         \\
                    & $[\hspace{0.5em}]$      & Indexierung                 &                         \\
                    & $->\quad.$              & Elementzugriff              &                         \\
                    & (Typ)\{init list\}      & compound literal (C99)      &                         \\ \hline
\multirow{7}{*}{14} & $++\quad--$(Präfix)     & Präfix-Ink/dekrement        & \multirow{7}{*}{R - L}  \\
                    & $+\quad-$ (Vorzeichen)  & Vorzeichen                  &                         \\
                    & $!\quad\sim$            & logisches / bitweises NICHT &                         \\
                    & \&                      & Adresse                     &                         \\
                    & *                       & Zeigerdereferenzierung      &                         \\
                    & (Typ)                   & Typumwandlung(cast)         &                         \\
                    & sizeof                  & Speichergröße               &                         \\ \hline
\multirow{2}{*}{13} & $*\quad/$               & Multiplikation, division    & \multirow{2}{*}{L - R}  \\
                    & \%                      & Modulo                      &                         \\ \hline
12                  & $\quad-$                & Addition, Subtraktion       & L - R                   \\ \hline
11                  & \textless{}\textless{}  & Links/rechts-Shift          & L - R                   \\ \hline
10                  & $<\quad<=\quad>=\quad>$ & kleiner / grösser (gleich)  & L - R                   \\ \hline
9                   & $!=\quad==$             & (un)gleich                  & L - R                   \\ \hline
8                   & \&                      & bitweises UND               & L - R                   \\ \hline
7                   & \textasciicircum{}      & bitweises exklusives ODER   & L - R                   \\ \hline
6                   & $\vert$                 & bitweises ODER              & L - R                   \\ \hline
5                   & \&\&                    & logisches UND               & L - R                   \\ \hline
4                   & $\vert\vert$            & logisches ODER              & L - R                   \\ \hline
3                   & ?:                      & Bedingung /mini if          & R - L                   \\ \hline
\multirow{3}{*}{2}  & =                       & Zuweisung                   & \multirow{3}{*}{R - L}  \\
                    & 
                    \begin{tabular}{ccccc} 
                        *= & /= & \%= & += & -=\\ 
                    \end{tabular} 
                    &
                    \multirow{2}{*}{\begin{tabular}[c]{@{}l@{}}Zusammengesetzte\\Zuweisungen\end{tabular}}&\\ 
                    &
                    \begin{tabular}{ccccc}
                        \&= & \textasciicircum{}= & $\vert$= & \textless{}\textless{}= & \textgreater{}\textgreater{}= \\ 
                    \end{tabular} &  & \\ \hline
1                    & ,                      & Komma-Operator              & L - R                  \\ \hline
\end{tabular}%
}
} 

\subsubsection{Sizeof}

Sizeof gibt die Grösse(in Bytes) eines Datentyp/array's zurück.

\begin{lstlisting}[language = c]
sizeof(char);//immer 8
sizeof(myArray)/sizeof(myArray[0]);//liefert die groesse eines Arrays
\end{lstlisting}

\subsubsection{Mini if}

Der ternäre Operator braucht 3 Argumente. Je nach Wahrheitswert wird, nach links oder rechts gesprungen.
\begin{lstlisting}[language = c]
<Bedinging>?"wahr":"unwahr";

return myInt==3?5:0;//Wenn myInt ist 3, wird 5 dem return Operator uebergeben. Ansonsten wird 0 uebergeben
\end{lstlisting}

